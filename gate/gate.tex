\documentclass[12pt,-letter paper]{article}

%\usepackage[left=1.5in,right=1in,top=1in,bottom=1in]{geometry}
%\usepackage[left=1.5in,right=1in]{geometry}
%\usepackage{geometry}
%\makeatletter%
%\textheight     243.5mm
%\textwidth      183.0mm
%\textwidth=31pc%
%\textheight=48pc
\usepackage{lipsum}% this package is included to get dummy paragraphs for sample purpose.
\usepackage{ulem}
\usepackage{alltt}
\usepackage{tfrupee}
\usepackage[anticlockwise,figuresright]{rotating}
\usepackage{pstricks}
\usepackage{wrapfig}
\usepackage{graphicx}
\usepackage{pstcol,pst-grad}
 \usepackage{bm}
\usepackage{enumitem}
\usepackage{circuitikz}
\usepackage{listings}
    \usepackage{color}                                            %%
    \usepackage{array}                                            %%
    \usepackage{longtable}                                        %%
    \usepackage{calc}                                             %%
    \usepackage{multirow}                                         %%
    \usepackage{hhline}                                           %%
    \usepackage{ifthen}                                           %%
  %optionally (for landscape tables embedded in another document): %%
    \usepackage{lscape}     
    \usepackage{gensymb}     
    \usepackage{tabularx}
\usepackage{ifthen}%
\usepackage{amsmath}%
\usepackage{color}%
\usepackage{float}%
\usepackage{graphicx}%
%\usepackage[right]{showlabels}%
\usepackage{boites}%
\usepackage{boites_exemples}%
\usepackage{graphicx,pstricks}
%\usepackage{enumerate}%
\usepackage{latexsym}
\usepackage[fleqn]{mathtools}
\usepackage{amssymb}
\usepackage{amssymb,amsfonts,amsthm}
\usepackage{mathrsfs,makeidx,listings,verbatim,moreverb}
\usepackage{siunitx}
%%\usepackage{amsthm,mathrsfs,makeidx,listings,verbatim,moreverb}
%\let\eqref\ref%  updated on 20th April 2017

\usepackage{hyperref}%
%\usepackage[dvips]{hyperref}%
\hypersetup{bookmarksopen=false}%
\usepackage{breakurl}%
\usepackage{tkz-euclide} % loads  TikZ and tkz-base
\DeclarePairedDelimiter\abs{\lvert}{\rvert}

\newcommand{\solution}{\noindent \textbf{Solution: }}
\providecommand{\mbf}{\mathbf}
\providecommand{\rank}{\text{rank}}
%\providecommand{\pr}[1]{\ensuremath{\Pr\left(#1\right)}}
\providecommand{\qfunc}[1]{\ensuremath{Q\left(#1\right)}}
\providecommand{\sbrak}[1]{\ensuremath{{}\left[#1\right]}}
\providecommand{\lsbrak}[1]{\ensuremath{{}\left[#1\right.}}
\providecommand{\rsbrak}[1]{\ensuremath{{}\left.#1\right]}}
\providecommand{\brak}[1]{\ensuremath{\left(#1\right)}}
\providecommand{\lbrak}[1]{\ensuremath{\left(#1\right.}}
\providecommand{\rbrak}[1]{\ensuremath{\left.#1\right)}}
\providecommand{\cbrak}[1]{\ensuremath{\left\{#1\right\}}}
\providecommand{\lcbrak}[1]{\ensuremath{\left\{#1\right.}}
\providecommand{\rcbrak}[1]{\ensuremath{\left.#1\right\}}}
\newenvironment{amatrix}[1]{%
  \left(\begin{array}{@{}*{#1}{c}|c@{}}
}{%
  \end{array}\right)
}
\theoremstyle{remark}
\newtheorem{rem}{Remark}
\newtheorem{theorem}{Theorem}[section]
\newtheorem{problem}{Problem}
\newtheorem{proposition}{Proposition}[section]
\newtheorem{lemma}{Lemma}[section]
\newtheorem{corollary}[theorem]{Corollary}
\newtheorem{example}{Example}[section]
\newtheorem{definition}[problem]{Definition}
\newcommand{\sgn}{\mathop{\mathrm{sgn}}}
%\providecommand{\abs}[1]{\left\vert#1\right\vert}
%\providecommand{\res}[1]{\Res\displaylimits_{#1}} 
%\providecommand{\norm}[1]{\left\lVert#1\right\rVert}
%\providecommand{\norm}[1]{\lVert#1\rVert}
\providecommand{\mtx}[1]{\mathbf{#1}}
%\providecommand{\mean}[1]{E\left[ #1 \right]}
\providecommand{\fourier}{\overset{\mathcal{F}}{ \rightleftharpoons}}
%\providecommand{\hilbert}{\overset{\mathcal{H}}{ \rightleftharpoons}}
\providecommand{\system}{\overset{\mathcal{H}}{ \longleftrightarrow}}
	%\newcommand{\solution}[2]{\textbf{Solution:}{#1}}
%\newcommand{\solution}{\noindent \textbf{Solution: }}
\newcommand{\cosec}{\,\text{cosec}\,}
\providecommand{\dec}[2]{\ensuremath{\overset{#1}{\underset{#2}{\gtrless}}}}
\newcommand{\myvec}[1]{\ensuremath{\begin{pmatrix}#1\end{pmatrix}}}
\newcommand{\myaugvec}[2]{\ensuremath{\begin{amatrix}{#1}#2\end{amatrix}}}
\newcommand{\mydet}[1]{\ensuremath{\begin{vmatrix}#1\end{vmatrix}}}
\newcommand\figref{Fig.~\ref}
\newcommand\appref{Appendix~\ref}
\newcommand\tabref{Table~\ref}
\newcommand{\romanNumeral}[1]{\uppercase\expandafter{\romannumeral#1}}
%\newcommand{\pr}[1]{\mathbb{P}(#1)}
%\numberwithin{equation}{section}
%\numberwithin{equation}{subsection}
%\numberwithin{problem}{section}
%\numberwithin{definition}{section}
%\makeatletter
%\@addtoreset{figure}{problem}
%\makeatother

%\let\StandardTheFigure\thefigure
\let\vec\mathbf
\def\inputGnumericTable{}                                 %%
%New macro definitions
\newcounter{matchleft}\newcounter{matchright}

\newenvironment{matchtabular}{%
  \setcounter{matchleft}{0}%
  \setcounter{matchright}{0}%
  \tabularx{\textwidth}{%
    >{\leavevmode\hbox to 1.5em{\stepcounter{matchleft}\arabic{matchleft}.}}X%
    >{\leavevmode\hbox to 1.5em{\stepcounter{matchright}\alph{matchright})}}X%
    }%
}{\endtabularx}
\title{Assignment}
\date{\today}
\begin{document}
\maketitle{Gate Question}

  \begin{enumerate}
  
      \item Given below is the diagram of a synchronous sequential circuit with one $J-K$ flip-flop and one $T$ flip-flop with their outputs denoted as $A$ and $B$ respectively, with $J_{A}=\brak{A^{\prime}+B^{\prime}}$, $K_{A}=(A+B)$ and $T_{B}=A$.\\
     
	      \begin{figure}[H]
		      \centering
		      
        \begin{circuitikz}
            \draw (2,2)coordinate(w)--(5,2)coordinate(x)--(5,6)coordinate(x)--(2,6)coordinate(z)--(2,2)coordinate(w);
            \draw (1,5.5)--(2,5.5);
           \draw (1,5.8)node[]{$A^{\prime}+B^{\prime}$};
            \draw (2,5.5)node[right]{$J_A$};
            \draw (5,5.5)node[left]{$A$};
            \draw (5,5.5)--(5.5,5.5);
            \draw (1,2.5)--(2,2.5);
            \draw (1,2.8)node[]{$A+B$};
            \draw(2,2.5)node[right]{$K_A$};
            \draw(5,2.5)node[left]{$A^{\prime}$};
           \draw(5,2.5)--(5.5,2.5);
           \draw[->] (0,4)--(2,4);
            \draw (0,4)--(0,0);
            \draw (-1,0)--(7,0);
            \draw (0,0.3)node[left]{$\text{clock}$};
            \draw  (2,4.5)--(2.5,4)--(2,3.5);
            
             \draw (8,2)rectangle(10,6); 
             \draw (8,4.5)--(8.3,4)--(8,3.5);
             \draw (7,5.5)--(8,5.5)node[right]{$T_{B}$};
             \draw (10,5.5)node[left]{$B$}--(10.5,5.5);
             \draw (10,2.5)node[left]{$B^{\prime}$}--(10.5,2.5);
             \draw[->] (7,4)--(8,4);
             \draw (7,4)--(7,0);
            
        \end{circuitikz}

		      \caption{synchronous sequence circuit of J-K and T flipflop}
		      \label{fig}
	      \end{figure} 








    Starting from the initial state $\brak{AB=00}$, the sequence of states $\brak{AB}$ visited by the circuit is
		  \hfill{gate IN2021,36}
		  \begin{enumerate}
          \item $00 \rightarrow 01 \rightarrow 10 \rightarrow 11 \rightarrow 00 
          \dots $
          \item $00 \rightarrow 10 \rightarrow 01 \rightarrow 11 \rightarrow 00 
          \dots $
          \item $00 \rightarrow 10 \rightarrow 11 \rightarrow 01 \rightarrow 00 \dots $
          \item $00 \rightarrow 01 \rightarrow 11 \rightarrow 00 \dots $
      \end{enumerate}
  \end{enumerate}
  
\end{document}
