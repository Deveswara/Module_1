\documentclass[12pt,-letter paper]{article}
\usepackage{siunitx}
\usepackage{setspace}
\usepackage{gensymb}
\usepackage{xcolor}
\usepackage{caption}
%\usepackage{subcaption}
\doublespacing
\singlespacing
\usepackage[none]{hyphenat}
\usepackage{amssymb}
\usepackage{relsize}
\usepackage[cmex10]{amsmath}
\usepackage{mathtools}
\usepackage{amsmath}
\usepackage{commath}
\usepackage{amsthm}
\interdisplaylinepenalty=2500
%\savesymbol{iint}
\usepackage{txfonts}
%\restoresymbol{TXF}{iint}
\usepackage{wasysym}
\usepackage{amsthm}
\usepackage{mathrsfs}
\usepackage{txfonts}
\let\vec\mathbf{}
\usepackage{stfloats}
\usepackage{float}
\usepackage{hyperref}
\usepackage{cite}
\usepackage{cases}
\usepackage{subfig}
%\usepackage{xtab}
\usepackage{longtable}
\usepackage{multirow}
%\usepackage{algorithm}
\usepackage{amssymb}
%\usepackage{algpseudocode}
\usepackage{enumitem}
\usepackage{mathtools}
%\usepackage{eenrc}
%\usepackage[framemethod=tikz]{mdframed}
\usepackage{listings}
%\usepackage{listings}
\usepackage[latin1]{inputenc}
%%\usepackage{color}{   
%%\usepackage{lscape}
\usepackage{textcomp}
\usepackage{titling}
\usepackage{hyperref}
%\usepackage{fulbigskip}   
\usepackage{tikz}
\usepackage{graphicx}
%\usepackage[left=1in, right=2in, top=1in, bottom=1in]{geometry}

\lstset{
  frame=single,
  breaklines=true
}
\let\vec\mathbf{}
\usepackage{enumitem}
\usepackage{graphicx}
\usepackage{siunitx}
\let\vec\mathbf{}
\usepackage{enumitem}
\usepackage{graphicx}
\usepackage{enumitem}
\usepackage{tfrupee}
\usepackage{amsmath}
\usepackage{amssymb}
\usepackage{mwe} % for blindtext and example-image-a in example
\usepackage{wrapfig}
\graphicspath{{figs/}}
\providecommand{\mydet}[1]{\ensuremath{\begin{vmatrix}#1\end{vmatrix}}}
\providecommand{\myvec}[1]{\ensuremath{\begin{bmatrix}#1\end{bmatrix}}}
\providecommand{\cbrak}[1]{\ensuremath{\left\{#1\right\}}}
\providecommand{\brak}[1]{\ensuremath{\left(#1\right)}}
\providecommand{\norm}[1]{\left\lVert#1\right\rVert}
\providecommand{\abs}[1]{\left\vert#1\right\vert}

\title{Assignment}
\date{\today}

\begin{document}

\maketitle{Questions}

\begin{enumerate}
\section{Vectros}
      \item Let $\overrightarrow{a}=\hat{i}+2\hat{j}-3\hat{k}$ and $\overrightarrow{b}=3\hat{i}-\hat{j}+2\hat{k}$ be two vector. Show that vector $\brak{\overrightarrow{a}+\overrightarrow{b}}$ and $\brak{\overrightarrow{a}-\overrightarrow{b}}$ are perpendicular to each other. 

      \item $X$ and $Y$ are two points with position vectors $3\overrightarrow{a}+\overrightarrow{b}$ and $\overrightarrow{a}-3\overrightarrow{b}$ respectively. Write the position vector of a point $Z$ which divides the line segment $XY$ in the ratio $2:1$ externally.
    
\section{Linear Forms}
    \item Find the acute angle between the planes 
        \begin{align*}
         \overrightarrow{r}.\hspace{6pt}\brak{\hat{i}-2\hat{j}-2\hat{k}}=1
         \end{align*}
      and 
        \begin{align*}
          \overrightarrow{r}.\hspace{6pt}\brak{3\hat{i}-6\hat{j}+2\hat{k}}=0.
         \end{align*}

    \item Find the length of the intercept, cut off by the plane $2x+y-z=5$ on the x-axis.


\section{Differentiation}
    \item Form the differential equation representing the family of curves $y=\frac{A}{x}+5$, by eliminating the arbitrary constant $A$.

    \item If $y=\csc{\brak{\cot{\sqrt{x}}}}$, then find $\dfrac{dy}{dx}$.

    \item If $y=\log(\cos{e^x})$, then find $\dfrac{dy}{dx}$.

    \item Solve the following differential equation :
     \begin{align*}
         \dfrac{dy}{dx}+y=\cos{x} - \sin{x}.
     \end{align*}

    \item Write the integrating factor of the differential equation
    \begin{align*}
    \brak{\tan^{-1}{y-x}}dy=\brak{1+y^{2}}dx.
    \end{align*}

    \item If $x = \sin{t}$, $y = \sin{pt}$, prove that
    \begin{align*}
    \brak{1-x^{2}}\dfrac{d^{2}x}{dx^{2}}-x\dfrac{dy}{dx}+p^{2}y=0.
    \end{align*}

    \item Solve the differential equation $\dfrac{dy}{dx}=1+x^{2}+y^{2}+x^{2}y^{2}$, given that $ y = 1 $ when $x = 0$.

    \item Find the particular solution of the differential equation $\dfrac{dy}{dx}=\frac{xy}{x^2+y^2}$, given that $y=1$ when $x=0$.

    \item Solve the following  equation :
     \begin{align*}
         \dfrac{dx}{dy}+x=\brak{\tan{y}+\sec^{2}{y}}.
     \end{align*}
 
 
\section{Martrices}
     \item If $A$ is a square matrix of order $3$, with $\abs{A} = 9$, then write the value of $\abs{2.\hspace{6pt}\text{adj}A}$.

     \item If $A$ and $B$ are symmertic matrices, such that $AB$ and $BA$ are both defined, then prove that $AB-BA$ is a skew matrix.

    \item Using properties of determinants, find the value of $x$ for which
    \begin{align*}
        \mydet{4-x&4+x&4+x\\4+x&4-x&4+x\\4+x&4+x&4-x}=0.
    \end{align*}
    
\section{Integration}
     \item Find :
        \begin{align*}
         \int_{-\dfrac{\pi}{4}}^{0}  \frac{1+\tan{x}}{1- \tan{x}} \,dx
        \end{align*}

     \item Find :
        \begin{align*}
            \int x.\hspace{6pt}\tan^{-1}{x} \,dx.
        \end{align*}

    \item Find :
        \begin{align*}
            \int \dfrac{dx}{\sqrt{5-4x-2x^2}}.
        \end{align*}

    \item Integrate the function
    \begin{align*}
    \frac{\cos{\brak{x+a}}}{\sin{\brak{x+b}}}
    \end{align*}
    w. r. t. $x$.
        
\section{Function}
    \item Let $*$ be an operation defined as $*$: $\textbf{R}\times\textbf{R}\rightarrow \textbf{R} $ such that $a * b = 2a + b,\hspace{6pt}a$, $b\in \textbf{R} $ . Check if * is a binary operation. If yes, find if it is associative too.

    \item Let $A$ = $R-\cbrak{2}$ and $B=R\cbrak{1}$. If $f: A\overrightarrow{}  B$ is a function defined by $f\brak{x}$=$\dfrac{x-1}{x-2}$, show that $f$ is one-one and onto. Hence, find $f^{-1}$.

    \item Show that the relation $S$ in the set $A = \cbrak{x \in Z : 0 \leq x \leq 12}$ given by $S = \cbrak{(a,\hspace{6pt}b) : a,\hspace{6pt}b \in Z,\hspace{6pt}\abs{a-b}}$ is divisible by $3$ is an equivalence relation.

    \item Let $* : N \times N$ $\rightarrow N $ be an operation defined as $a * b = a + ab, \hspace{6pt}\forall a,\hspace{6pt}b \in N $. Check if * is a binary operation. If yes, find if it is associative too.


\section{Probability}
    \item $12$ cards numbered $1$ to $12$ (one number on one card), are placed in a box and mixed up thoroughly. Then a card is drawn at random from the box. If it is known that the number on the drawn card is greater than $5$, find the probability that the card bears an odd number.

    \item Out of $8$ outstanding students of a school, in which there are $3$ boys and $5$ girls, a team of $4$ students is to be selected for a quiz competition. Find the probability that $2$ boys and $2$ girls are selected.

    \item In a multiple choice examination with three possible answers for each of the five questions, what is the probability that a candidate would get four or more correct answers just by guessing ?

    \item An insurance company insured $3000$ cyclists,  $6000$ scooter drivers and $9000$ car drivers. The probability of an accident involving a cyclist, a scooter driver and a car driver are $0.3$, $0.05$ and $0.02$ respectively. One of the insured persons meets with an accident. What is the probability that he is a cyclist ?
    
\end{enumerate}
\end{document}
